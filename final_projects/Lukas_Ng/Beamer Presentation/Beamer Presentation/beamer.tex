\documentclass{beamer}
\usepackage[latin1]{inputenc}
\usepackage{graphicx}
\usetheme{Warsaw}

\title{Perfect Numbers}
\author{Lukas Ng}\institute{University of Washington}

\begin{document}

\begin{frame}
\titlepage

\end{frame}

\begin{frame}
What is a Perfect Number?

\bigskip

What a great question audience!

\end{frame}

\begin{frame}

A perfect number is an integer $n$, such that the sum of its divisors are equal to $2n$.

The sum of divisor function is usually represented by $\sigma(n)$, and so we can say that if $n$ is a perfect number, then 

\begin{center}

$\sigma(n) = 2n$

\end{center}

\end{frame}

\begin{frame}

Here is a short list of some perfect numbers.

$6, 28, 496, 8128, 33550336, ...$

As we can see, the list of perfect numbers increases quite quickly. 

Thus far, there are $48$ known perfect numbers.

It should also be noted that the list of known perfect numbers only include even integers.

\end{frame}

\begin{frame}

\textbf{Mersenne Primes}

\bigskip

Prime numbers of the form: $m = 2^p - 1$, where both $m, p$ are primes.

Usually, the largest known prime is a Mersenne Prime.

\bigskip

But why do we care?

\end{frame}

\begin{frame}

It turns out that Even Perfect Numbers can only be of the form:

\begin{center}

$2^{p-1} (2^p - 1)$

\end{center}

This was proven by Euclid.

\medskip

That second part looks awfully familiar... (Hint: It looks a lot like a Mersenne Prime)

\medskip

So now let's take a look at the proof of this fact.

\end{frame}

\begin{frame}

Just kidding, it is a bit long for this presentation. For now, we will just accept this as fact. See page 6 and 7 of my paper for the proof.

\end{frame}

\begin{frame}

Now, we will turn our attention to Odd Perfect Numbers.

\bigskip

Thus far, no Odd Perfect Number has ever been discovered. 

It is widely believed by mathematicians that no Odd Perfect Numbers exist.

I wrote some short code presented in my paper that can perhaps convince us of this fact.

\bigskip

That does not mean we know nothing about these numbers though...

\end{frame}

\begin{frame}

It turns out that while mathematicians have been unable to prove Odd Perfect Numbers do not exist, they have been able to prove various necessary conditions for them to exist.

\end{frame}

\begin{frame}

Here are a few conditions necessary for the existence of Odd Perfect Numbers:

\bigskip

- An OPN must have at least $9$ distinct prime divisors.

- If an OPN is not divisible by $3$, it must have at least $12$ distinct prime divisors.

- If an OPN is not divisible by $3$ or $5$, it must have at least $15$ distinct prime divisors.

- If an OPN is not divisible by $3$, $5$, or $7$, it must have at least $27$ distinct prime divisors.

- Any OPN must have a prime factor larger than $10^8$.

- And more...

\end{frame}

\begin{frame}

This problem has been around for quite some time and worked on by a lot of mathematicians.

To date, it has been verified that no OPNs exist up to $10^{1500}$.

\bigskip

A proof of this could come at any day, or escape us for the rest of our individual lives. 

Either way, a seemingly trivial problem like this shows us that there is still much for us to learn and discover about numbers.

\end{frame}

\end{document}
%sagemathcloud={"zoom_width":90}