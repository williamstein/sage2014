\documentclass{beamer}
\usepackage[latin1]{inputenc}
\usepackage{graphicx}
\usetheme{Warsaw}
\setlength{\parskip}{1em}

\title{Curves, Surfaces of Revolution and Surfaces of Two Variables in Differential Geometry}
\author{Dylan Estes and Evan Sadler}\institute{University of Washington}

\begin{document}

\begin{frame}
\titlepage

\end{frame}

\begin{frame}
The point of this project is to give an accurate and interactive model from which new students to Differential Geometry can play around with until they become more comfortable with the subject.
\end{frame}

\begin{frame}
Definition: A parametrized differential curve is a differentible map $\alpha: I \rightarrow R^3$ of an open interval $I = (a,b)$ of the real line $R$ into $R^3$

Example: $\alpha(t) = (acos(t),asin(t),bt)$,  $t\in R$
\end{frame}

\begin{frame}
Important Vectors:\\
Tangent: $t = \alpha'(s)$\\
Normal: $n = \alpha''(s)/\mid k(s)\mid$\\
Binormal: $b$ = $t$ x $n$
\end{frame}


\begin{frame}
Torsion: 
Definition: Let $\alpha:I \rightarrow R^3$ be a curve parametrized by arc length s such that $\alpha''(s) \neq 0, s \in I$. The number $\tau (s)$ defined by $ b'(s) = \tau (s)n(s)$ is called the torsion of $\alpha$ at s

The torsion of a curve measures how sharply it is twisting out of the plane of curvature
\end{frame}

\begin{frame}
Curvature:
Let $\alpha:I \rightarrow R^3$ be a curve parameterized by arc length $s \in I$. The number $\mid \alpha''(s)\mid = k(s)$ is called the curvature of $\alpha$ at s.

$\mid \alpha''(s)\mid$ therefore, is a measure of how rapidly the curve pulls away from the tangent line at s in a neighborhood of s
\end{frame}

\begin{frame}
\includegraphics[width = 90mm]{helix.png}
\end{frame}

\begin{frame}[t]
\begin{center}
\Huge TIME FOR SURFACES
\end{center}

\begin{figure}[ht!]
\centering
\includegraphics[width=90mm]{space.jpg}
\caption{}
\label{overflow}
\end{figure}
\end{frame}

\begin{frame}
\begin{center}
Forget $R^{3}$! 

Gauss proved that curvature does not depend on how surfaces are embedded in 3-dimensional space.

Used in general theory of relativity. 
\end{center}
\end{frame}

\begin{frame}[t]
\begin{center}
\Huge EXAMPLE TIME
\end{center}
\begin{figure}[ht!]
\centering
\includegraphics[width=70mm]{Donut_Guy.png}
\caption{}
\label{overflow}
\end{figure}

\end{frame}

\end{document}


%sagemathcloud={"zoom_width":105}