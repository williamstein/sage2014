\documentclass{beamer}
\usepackage[latin1]{inputenc}
\usepackage{graphicx}
\usetheme{Dresden}

\title{Markov chain class in Sage}
\author{Sam Michels}\institute{Math 480B}

\begin{document}

\begin{frame}
\titlepage

\end{frame}

\begin{frame}
\frametitle{What is a Markov chain?}
A Markov chain is a mathematical system that undergoes transitions from one state to another within a "state space". Markov chains are random processes, and are memoryless, meaning the probability of going to some state from the current state depends solely on the current state and not any of the preceding states.
\end{frame}

\begin{frame}
\frametitle{Why use Markov chains?}
Markov chains generally have a finite or countable state space, and because of their memorylessness Markov chains see a wide variety of applications from biology and sociology to economics and the stock market
\end{frame}


\begin{frame}
\frametitle{Example}
Let a person be in state P when the last cola a person has drank was a Pepsi and C when it was a coke. Suppose the probability that a person who last drank coke will next drink coke is 80\%, and the probability that a person who last drank pepsi will next drink pepsi is 90\%. Then we have a Markov chain with the following transition matrix (ill put a transition matrix in later). (After that (when I figure how to bring code into latex) ill show some properties /etc using my MC class) 
\end{frame}

\end{document}
%sagemathcloud={"zoom_width":90}